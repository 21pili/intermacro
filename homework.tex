\documentclass{article}
\usepackage[utf8]{inputenc}
\usepackage[T1]{fontenc}
\usepackage[french]{babel}
\usepackage{amsmath, amssymb, amsthm}
\usepackage{geometry}
\usepackage{titling}
\usepackage{fancyhdr}
\usepackage{lipsum}
\usepackage{parskip}
\usepackage{forest}
\usepackage{tikz}
\usepackage{stmaryrd}
\usepackage{listings}
\usepackage{graphicx}
\usepackage{float}

\geometry{top=4cm, bottom=4cm, left=4cm, right=4cm}
\pagestyle{fancy}
\fancyhf{}
\rhead{Pierre Pili}
\lhead{International Macroeconomics - Homework 1}
\cfoot{\thepage}

\title{INTERNATIONAL MACRO - HOMEWORK 1}
\author{Pierre Pili}
\date{\today}

\begin{document}

\begin{titlingpage}
\maketitle
\end{titlingpage}

\tableofcontents

\newpage
\section{A 3-Period Model}
\subsection{}
One can derive the following budget constraints from a 3-period small open endowment economy :
\begin{alignat}{2}
    C_1 + B_1 - B_0 &= Y_1 + r_0 B_0  \quad&\\
    C_2 + B_2 - B_1 &= Y_2 + r^* B_1  \quad&\\
    C_3 + B_3 - B_2 &= Y_3 + r^* B_2  \quad&
  \end{alignat}
Where $B_t$ and $Y_t$ are respectively the asset position and the endowment of households at period $t$.
\subsection{}
In a 3-period model, agents will not accumulate bonds in period three as they will not make it to the next period. They will rather maximize their consumption. Which implies $B_3 = 0$
\subsection{}
Rewriting (3) with the transversality condition :
\begin{alignat}{2}
    B_2 &= \frac{C_3 - Y_3}{1+r^*}  \quad&
\end{alignat}
Inserting the expression of $B_2$ in (2) :
\begin{alignat}{2}
    C_2 + \frac{C_3 - Y_3}{1+r^*} &= Y_2 + (1+r^*)B_1  \quad&
\end{alignat}
Which implies the following expression for $B_1$ :
\begin{alignat}{2}
    B_1 &= \frac{C_2 - Y_2}{1+r^*} + \frac{C_3 - Y_3}{(1+r^*)^2}  \quad&
\end{alignat}
Inserting the expression of $B_1$ in (1) :
\begin{alignat}{2}
    C_1 + \frac{C_2 - Y_2}{1+r^*} + \frac{C_3 - Y_3}{(1+r^*)^2} &= Y_1 + (1+r_0)B_0  \quad&
\end{alignat}
Rearranging (7) :
\begin{alignat}{2}
    C_1 + \frac{C_2}{1+r^*} + \frac{C_3}{(1+r^*)^2} &= (1+r_0)B_0 + Y_1 + \frac{Y_2}{1+r^*} + \frac{Y_3}{(1+r^*)^2} \equiv W \quad&
\end{alignat}
Where $W$ denotes the intertemporal wealth of households. $W$ is exogenous in this model since $Y_t, r_0, r^*$ and $B_0$ are exogenous.
\subsection{}
Households want to maximize their lifetime utiliy subject to the IBC. We thus form the lagrangian of this maximization problem :
\begin{alignat}{2}
    \mathcal{L}(C,\lambda) &= U(C) + \lambda \left[ W - \sum_t \frac{C_t}{(1+r^*)^{t-1}} \right] \quad&
\end{alignat}
Where $C$ is the vector of consumption $(C_1, C_2, C_3)$, $U$ the lifetime utility function, and where $\displaystyle\sum_t$ denotes the sum over all periods.\newline
The first order conditions imply, for $t \in$ $\{ 1, 2, 3\}$ :
\begin{alignat*}{2}
    \frac{\partial\mathcal{L}}{\partial C_t} (C,\lambda) = 0 &\iff C_t = \frac{(1+r^*)^{t-1}}{\lambda} \quad&
\end{alignat*}
Inserting the expressions for $C_t$ in the IBC implies :
\begin{alignat}{2}
    W = \sum_t \frac{1}{\lambda} &\iff \lambda = \frac{3}{W}\quad&
\end{alignat}
Which yields for $t \in$ $\{ 1, 2, 3\}$ :
\begin{alignat}{2}
    C_t &= \frac{(1+r^*)^{t-1}}{3}W\quad&
\end{alignat}
This result is similar to the 2-period economy with no discount factor for consumption in the lifetime utility.\newline
We can now easily compute the trade balance and the current account for each period :
\begin{alignat*}{2}
    TB_t &= Y_t - C_t\quad&\\
    &= Y_t - \frac{(1+r^*)^{t-1}}{3} \left[ (1+r_0)B_0 + Y_1 + \frac{Y_2}{1+r^*} + \frac{Y_3}{(1+r^*)^2} \right]\quad&\\
    &= \frac{(1+r^*)^{t-1}}{3} \left[\frac{3}{(1+r^*)^{t-1}}Y_t - (1+r_0)B_0 - Y_1 - \frac{Y_2}{1+r^*} - \frac{Y_3}{(1+r^*)^2} \right]\quad&
\end{alignat*}
Thus :
\begin{alignat}{2}
    TB_1 &= \frac{1}{3} \left[2 Y_1 -(1+r_0)B_0 - \frac{Y_2}{1+r^*} - \frac{Y_3}{(1+r^*)^2} \right]\quad&\quad&\\
    TB_2 &= \frac{(1+r^*)}{3} \left[\frac{2Y_2}{1+r^*} - (1+r_0)B_0 - Y_1 - \frac{Y_3}{(1+r^*)^2} \right]\quad&\\
    TB_3 &= \frac{(1+r^*)^2}{3} \left[\frac{2Y_3}{(1+r^*)^2} - (1+r_0)B_0 - Y_1 - \frac{Y_2}{(1+r^*)} \right]\quad&
\end{alignat}
\subsection{}
A positive endowment shock $\Delta Y_1$ will have a postive impact on the trade balance and on consumption in period 1, indeed, derivating (12) :
\begin{alignat}{2}
    \Delta TB_1 &= \Delta CA_1 = \frac{2}{3} \Delta Y_1  > 0 \quad&
\end{alignat}
Which implies,
\begin{alignat}{2}
    \Delta C_1 &= \frac{1}{3} \Delta Y_1  > 0 \quad&
\end{alignat}
\subsection{}
In a similar way, derivating (12) :
\begin{alignat}{2}
    \Delta TB_1 &= \Delta CA_1 = \Delta Y \left[2 - \frac{1}{1+r^*} - \frac{1}{(1+r^*)^2} \right] = \Delta Y \left[3 r^* + o(r^*)\right]\quad&
\end{alignat}
The trade balance in positively impacted by a permanent endowment shock but proportionnaly to $r^*$ at the first order. The impact will be very small.\newline
Thus permanent and transitory schocks have very different impact on the current account. While the former have a positive impact on the current account, the latter has almost no impact on it whatsoever. 
\begin{alignat*}{2}
    \Delta C_1 &= \Delta Y - \Delta TB_1 \quad&\\
    &= \Delta Y \left[1 - 3 r^* + o(r^*)\right]\quad&\\
\end{alignat*}
A permanent shock will increase consumption by a one-one coefficient at the first order with respect to $r^*$. 
\subsection{}
The way results were written so far very much suggests the general form of consumption in a N-period economy. The transversality condition writes $B_N = 0$.
In that case the intertemporal IBC writes :
\begin{alignat}{2}
    \sum_{t} \frac{C_t}{(1+r^*)^{(t-1)}} &= (1+r_0)B_0 + \sum_t \frac{Y_t}{(1+r^*)^{(t-1)}} \equiv W \quad&
\end{alignat}
The lagrangian is the same as in (9), which implies for any $t$ in $\llbracket 1, N \rrbracket$  that :
\begin{equation}{}
    C_t = \frac{(1+r^*)^{t-1}}{N}W 
\end{equation}
\begin{equation}{}
    TB_t = Y_t - \frac{(1+r^*)^{t-1}}{N} \left[ (1+r_0)B_0 - \sum_i \frac{Y_i}{(1+r^*)^{i-1}} \right]
    \label{tb_t}
\end{equation}
One is now able to redo question 1.5 and 1.6.\newline
\subsubsection*{1.5 bis}
Deriving \eqref{tb_t} for a positive endowment shock $\Delta Y_1$:
\begin{alignat*}{2}
    \Delta CA_1 = \Delta TB_1 &= \Delta Y_1 - \frac{1}{N} \Delta Y_1 \quad&\\
    &= \left[ 1 - \frac{1}{N}\right] \Delta Y_1 > 0 \quad&
    \label{Delta_tb}
\end{alignat*}
The trade balance in period 1 is positively impacted by a positive endowment shock in period 1. The variation in consumption writes :
\begin{alignat*}{2}
    \Delta C_1 &= \Delta Y_1 - \Delta TB_1 \quad&\\
    &= \frac{1}{N} \Delta Y_1 > 0 \quad&
\end{alignat*}
Consumption is also positively impacted by this transitory shock on endowment.
\subsubsection*{1.6 bis}
In a permanent shock situation, one can derive \eqref{tb_t} :
\begin{alignat*}{2}
    \Delta CA_1 = \Delta TB_1 &= \Delta Y \left[N - \sum_t \frac{1}{(1+r^*)^{t-1}} \right]\quad&\\
    &= \Delta Y \left[N - \sum_t \Big(1 - (t-1)r^* + o(r^*) \Big)\right] \quad&\\
    &= \Delta Y \left[(N-1)N r^* + o(r^*) \right]\quad&
\end{alignat*}
As before, consumption will be negligible if we consider that $N$ is pegged and that $r^*$ is arbitrarily close to zero.
\subsection{}
The lifetime utiliy function now becomes :
\begin{equation}
    U(C) = \sum_t \beta^{t-1} \ln C_t
\end{equation}
The IBC is identical as before, thus the lagrangian of this maximization problems writes :
\begin{equation}
    \mathcal{L}(C,\lambda) = U(C) + \lambda \left[ W - \sum_t \frac{C_t}{(1+r^*)^{t-1}} \right]
\end{equation}
The first order conditions imply, for $t \in \llbracket 1, N \rrbracket$ :
\begin{alignat*}{2}
    \frac{\partial\mathcal{L}}{\partial C_t} (C,\lambda) = 0 &\iff C_t = \frac{\beta^{t-1}(1+r^*)^{t-1}}{\lambda} = \frac{1}{\lambda}  \quad&
\end{alignat*}
Inserting the expressions for $C_t$ in the IBC implies :
\begin{alignat}{2}
    \lambda W = \sum_t \frac{1}{(1+r^*)^{t-1}} = \frac{(1+r^*)^N-1}{r^*}  \quad&
\end{alignat}
The expression of $\lambda$ one can get from the last expression allows to find the consumption at each period $t$ at the equilibrium :
\begin{alignat}{2}
    C_t =   \frac{r^*}{(1+r^*)^N-1} W \quad&
    \label{c_t}
\end{alignat}
One can notice that the condition $\beta(1+r^*) = 1$ means that the discount factor and the decreasing price of consumption will cancel out, meaning that consumption will be perfectly smoothed over time, put otherwise, $C_t$ does not depend on $t$.\newline
Let $\rho$ denote the ratio $\displaystyle \frac{r^*}{(1+r^*)^N-1}$.
\subsubsection*{1.5 ter} 
Derivating \eqref{c_t} for $t=1$, for a transitory endowment shock $\Delta Y_1$ :
\begin{alignat*}{2}
    \Delta C_1 &=   \rho \Delta W \quad&\\
    &=   \rho \Delta Y_1 \quad&
\end{alignat*}
Which implies :
\begin{alignat*}{2}
    \Delta CA_1 = \Delta TB_1 &= \Delta Y_1 - \Delta C_1 \quad&\\
    &=   (1 - \rho) \Delta Y_1 \quad&
\end{alignat*}
If $r^*$ is small enough, one can show that $\rho \approx \frac{1}{N}$. Those results are thus very similar to the previous case.
\subsubsection*{1.6 ter} 
Derivating \eqref{c_t} for $t=1$, for a permanent endowment shock $\Delta Y$ and recalling the definition of $\rho$ :
\begin{alignat*}{2}
    \Delta C_1 &=   \rho \Delta W \quad&\\
    &= \rho \Delta Y \sum_t \frac{1}{(1+r^*)^{t-1}}\\
    &= \Delta Y
\end{alignat*}
Which implies :
\begin{alignat*}{2}
    \Delta CA_1 = \Delta TB_1 &= \Delta Y - \Delta C_1 \quad&\\
    &=   0
\end{alignat*}
In the N-period, discounted lifetime utility framework, the current account cannot be affected at all by a permanent endowment shock.
\newpage










\section{Imperfect Capital Mobility and Crowding Out}
The situation described in this exercise can be summarized by the following equations :
\subsubsection*{Households}
The budget constraints at each period writes :
\begin{equation}
    C_1 + B_1^h = Y_1 - T_1
\end{equation}
\begin{equation}
    C_2  = \Pi_2 + (1+r_1) B_1^h - T_2
\end{equation}
Which implies the IBC for households :
\begin{equation}
    C_1 + \frac{C_2}{1+r_1}  = Y_1 - T_1 + \frac{\Pi_2 - T_2}{1+r_1} \equiv W^h
\end{equation}
Households want to maximize their utiliy subject to the IBC, which writes :
\begin{equation}
    U(C_1)  = \ln(C_1) + \beta \ln\Big((1+r_1)(W^h-C_1)\Big)
\end{equation}
The first order condition writes :
\begin{equation}
    C_1  = \frac{C_2}{\beta(1+r_1)}
    \label{2.c_1}
\end{equation}
Inserting \eqref{2.c_1} into the IBC yields :
\begin{alignat}{2}
    C_1  &= \frac{1}{1 + \beta} W^h\quad&\\
    C_2  &= \frac{\beta}{\beta + 1} (1+r_1) W^h \quad&
\end{alignat}
\subsubsection*{Firms}
Firms ought to maximize their profit in period 2, which is the difference between the income from production and the cost of borrowing $I_1$ : 
\begin{equation}
    \Pi_2(I_1, r_1)  = 6\sqrt{I_1} - (1+r_1) I_1
\end{equation}
One must notice that firms take $r_1$ as given as they do not decide on it.\newline
The first order condition from this maximization problem yields :
\begin{equation}
    I_1  = \frac{9}{(1+r_1)^2}
\end{equation}
\subsubsection*{Government}
The budget constraints at each period writes :
\begin{equation}
    G_1 + B_1^g = T_1
\end{equation}
\begin{equation}
    G_2  = T_2 + (1+r_1) B_1^g
\end{equation}
Which implies the IBC for households :
\begin{equation}
    G_1 + \frac{G_2}{1+r_1}  = T_1 + \frac{T_2}{1+r_1} \equiv W^g
\end{equation}
One must notice that consumption actually does not depend on the path $(T_1, T_2)$ but only on the government spendings, $(G_1, G_2)$, indeed the intertemporal wealth of housholds writes :
\begin{equation}
    W^h  = Y_1 - G_1 + \frac{\Pi_2 - G_2}{1+r_1}
\end{equation}
To find the net foreign asset position, one can compute the trade balance at period 1 since $B_0 = 0$.\newline
Thus, $B_1 - B_0 = B_1 = TB_1 = Y_1 - G_1 - C_1 - I_1$\newline
To answer all the following, I used the following script :
\begin{lstlisting}[language=Python, label=python_code]
    import numpy as np
    import matplotlib.pyplot as plt
    
    #CONSTANTES
    beta = 0.96
    rstar = 0.08
    p = 0.02
    Y = 20

    #FONCTIONS

    def I1(r1):
        return 9 * 1/(1+r1)**2

    def Pi(r1, I1):
        return 6 * np.sqrt(I1) - (1+r1) * I1
    
    def Wh(r1, I1, G):
        return Y - G[0] + (Pi(r1, I1) - G[1]) / (1 + r1)
    
    def B1(r1, G):
        f = 1 / (1 + beta)
        I = I1(r1)
        C1 = f * Wh(r1, I, G)
        return Y - G[0] - C1 - I
\end{lstlisting}
\subsection{}
Running the previous code with $G_1 = 1$ and $G_2 = 7$ and both possible interest rates :
\begin{lstlisting}[language=Python, label=python_code]
    #QUESTION 1
    G = [1,7]
    Bstar = B1(rstar,G)
    Bp = B1(rstar+p,G)
    print("Bstar=",Bstar)
    print("Bp=",Bp)

[1] Bstar = 0.9601914840010082
    Bp = 1.3199527744982298
\end{lstlisting}
In both situation, the trade balance is positive, thus $r^1 = r^*$ and one can compute :
\begin{alignat}{2}
    CA_1  = TB_1 &\approx 0.96&\\
    I_1 &\approx 7.72
\end{alignat}
\subsection{}
Running the previous code with $G_1 = 2$ and $G_2 = 7$ and both possible interest rates :
\begin{lstlisting}[language=Python, label=python_code]
    #QUESTION 2
    G = [2,7]
    Bstar = B1(rstar,G)
    Bp = B1(rstar+p,G)
    print("Bstar=",Bstar)
    print("Bp=",Bp)

[2] Bstar = 0.47039556563366247
    Bp = 0.8301568561308823
\end{lstlisting}
In both situation, the trade balance is positive, thus $r^1 = r^*$ and one can compute :
\begin{alignat}{2}
    CA_1  = TB_1 &\approx 0.47&\\
    I_1 &\approx 7.72
\end{alignat}
At the moment, since $r^1$ is the same as in question 1, governemnent expenditures do not crow out investment.
\subsection{}
Running the previous code with $G_1 = 4$ and $G_2 = 7$ and both possible interest rates :
\begin{lstlisting}[language=Python, label=python_code]
    #QUESTION 3
    G = [4,7]
    Bstar = B1(rstar,G)
    Bp = B1(rstar+p,G)
    print("Bstar=",Bstar)
    print("Bp=",Bp)

[3] Bstar = -0.5091962711010325
    Bp = -0.14943498060381089
\end{lstlisting}
In both situations, the trade balance is negative, thus $r^1 = r^* + p$ and one can compute :
\begin{alignat}{2}
    CA_1  = TB_1 &\approx -0.14&\\
    I_1 &\approx 7.44
\end{alignat}
With interest rates being higher, investment decreases, we thus observe a crowding out effect of government spendings in this framework.
\subsection{}
Running the previous code with $G_1 = 4$ and $G_2 = 7$ and $p = 0.04$ :
\begin{lstlisting}[language=Python, label=python_code]
    #QUESTION 4
    G = [4, 7]
    p = 0.04
    Bstar = B1(rstar, G)
    Bp = B1(rstar + p, G)
    print("Bstar=",Bstar)
    print("Bp=",Bp)

[4] Bstar = -0.5091962711010325
    Bp = 0.19018117451062366
\end{lstlisting}
In this situation, the sign of the trade balance depends on the interest rate. Consider the following results :
\begin{lstlisting}[language=Python, label=python_code]
    print(Pi(rstar, I1(rstar)))
    print(Pi(rstar + p, I1(rstar + p)))

[5] 8.33333333333333
    8.035714285714283
\end{lstlisting}
Firms are better off when they pay interest rate $r^*$ but they deteriorate the trade balance in doing so, resulting in a interest premium. There is no equilibrium in this situation.
\begin{figure}[H]
    \centering
    \includegraphics[width=0.7\textwidth]{media/profits_wrt_interest_rates.png}
    \label{fig:mon_graphique}
\end{figure}
Consider the previous graph, if interest rates are $0.08$, then firms invest to reach $\Pi_2(0.08)$, but at the cost of the trade balance, meaning the interest will actually be $0.12$, firms will thus invest to reach $\Pi_2(0.12)$, improving the trade balance, thus interest rates will come back to $0.08$, and so on.
\newpage





\section{Interest Rate Uncertainty}
\end{document}


